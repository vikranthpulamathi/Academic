\documentclass[a4paper, 12pt]{article}
%for aligning equations within tabular columns; for mathbf and dfrac
\usepackage[margin = 1in]{geometry}
\usepackage{amsmath, amsfonts}
\usepackage{mathtools}
\usepackage{float}
\usepackage{caption}
\usepackage{graphicx}
\usepackage{multirow}
\usepackage{xcolor}
\numberwithin{equation}{section}
\usepackage{hyperref}
\renewcommand{\arraystretch}{1.5}
%\renewcommand{\baselinestretch}{1.3}

\title{\textbf{A Reading Project on Algebraic Structures}}
\author{Vikranth Pulamathi}
\date{January 20, 2021}


\begin{document}
\maketitle
\vspace{13cm}
\begin{center}
Dept. of PG Physics\\
St. Joseph's College (Autonomous)
Bangalore, India - 560027
\end{center}


\thispagestyle{empty}
\cleardoublepage

\tableofcontents
\thispagestyle{empty}
\cleardoublepage

\newpage
\pagenumbering{arabic}
\begin{center} 
 {\huge \textbf{Algebraic Structures}}\\
 \vspace{1mm}
A Summary by Vikranth Pulamathi\\
\hfill Started: January 20, 2021\\
\hfill Completed: 2021
\end{center}

\section{Introduction}
A collection of family of things that share similar properties is simply called a \textit{Set}, and the things in the set are called \textit{elements} of that set. A Cartesian Product of two sets A and B is defined as
\begin{equation}
A \times B = \{ (a, b) | a \in A, b \in B \}
\end{equation}
The \textit{cardinality} of a set $A$ is the number of elements of $A$, denoted as $|A|$. The cardinality of the empty set $\phi$ is zero. The set of natural numbers $\mathbb{N}$ has inifinite cardinality.

For two sets $A$ and $B$ and their complements $A'$ and $B'$, the principle of Inclusion-Exclusion says
\begin{equation}
|A \cup B| = |A| + |B| - |A \cap B|
\end{equation}
and De-Morgan's Laws state
\begin{subequations}
 \begin{equation}
 (A \cup B)' = A' \cap B'
 \end{equation}
 \begin{equation}
 (A \cap B)' = A' \cup B'
 \end{equation}
\end{subequations}
Any subset $R$ of the Cartesian Product $A \times B$ defines a relation from $A$ to $B$. Then, for $(a,b) \in R$, the relation is denotes as $a R b$. There are several kinds of relations as follows:
\begin{enumerate}
 \item \textbf{Empty Relation} is when $R = \phi$
 \item \textbf{Universal Relation} is when every element of $A$ is related to every element in $B$
 \item \textbf{Identity Relation} is when every element of $A$ is related to itself, $I = \{ (a,a)| a \in A \}$
 \item \textbf{Inverse Relation} is when if $R = \{ (a,b)\} \in A \times B $ is a relation, then the inverse is $R^{-1} = \{ (b, a)| a,b \in R \}$
 \item \textbf{Reflexive Relation} iff every element of $A$ maps to itself, i.e.,
 \begin{equation}
 \forall a \in A, \quad aRa \in R
 \end{equation}
 \item \textbf{Symmetric Relation} iff 
 \begin{equation}
  \forall a,b \in R, \quad aRb = bRa
 \end{equation}
 \item \textbf{Transitive Relation} iff 
 \begin{equation}
  \forall a,b,c \in R, \quad aRb \quad \& \quad bRc \Rightarrow aRc
 \end{equation}
 \item \textbf{Equivalence Relation} when a given relation $R$ is all Reflexive, Symmetric and Transitive.
\end{enumerate}

A \textit{function}, or a \textit{mapping} is a relation between some input(called Domain) and output(called Range). 
Let $f: A \to B$ and $g: B \to C$. Then, the composition $\phi \psi$  is a mapping from $A$ to $C$ defined as 
\begin{equation}
(f \circ g)(x) = f(g(x)), \quad \forall x \in A
\end{equation}
Functions are of the following types:
\begin{enumerate}
 \item \textbf{One to One} or \textbf{Injective Function}: A function $f: A \to B$ (or $f(a) = b$, $a \in A, b \in B$)is one-to-one if
 \begin{equation}
 f(a_1) = f(a_2) \Rightarrow a_1 = a_2, \quad \forall a_1, a_2 \in A
 \end{equation}
 
 \item \textbf{Many to One Function}: when one or more elements of $A$ maps to the same element in $B$
 
 \item \textbf{Onto} or \textbf{Surjective Function}: A function in which every element of $B$ has pre-image in $A$.
 
 \item \textbf{One-One Correspondence} or \textbf{Bijective Function}: If a function is both injective and surjective, i.e., if every element in $A$ has a unique element in $B$ AND every element of $B$ has a pre-image in $A$.
 
\end{enumerate}

A \textit{partition} of a set $S$ is defined as a collection of non-empty and disjoint subsets of $S$, whose union is the whole set $S$. The equivalence classes of an equivalence relation on a set $S$ constitute a partition of $S$. Conversely, for any partition $P$ of $S$, there is an equivalence relation on $S$ whose equivalence classes are the elements of $P$.


\newpage
\section{Group Like}
\subsection{Groups}

\textbf{Definition}: A \textit{Binary Operation} on a set $G$ is a function that assigns each ordered pair of $G$ an element of $G$.

So, if the memebers of the ordered pair of $G$ undergo a binary operation to produce another element of the same set $G$, then the corresponding binary operation is said to be \textit{closed}, and this is called the \textit{closure property} of a binary operation.

\noindent
\textbf{Definition:} Consider a set $G$ with a binary operation *(usually multiplication). Then, the set $G$ is called a \textit{Group} if it satisfies the following:
\begin{enumerate}
 \item \textbf{Associativity}: The binary operation must be associative, i.e.,
 \begin{equation}
 (a * b) * c = a * (b * c), \quad \forall a,b,c \in G
 \end{equation}
 \item \textbf{Existence of Identities:}
 \begin{equation}
 \exists \,e \in G \ni a*e = e*a = a, \forall a \in G
 \end{equation}
 \item \textbf{Existence of Inverse}
 \begin{equation}
 \forall a \in G, \exists \,b \in G \ni a*b = b*a = e
 \end{equation}
\end{enumerate}

\subsubsection{Properties of Groups}
\textbf{Theorem 2.1}: In a group $(G, *)$, there is only one identity element\\
\textbf{Proof:} Suppose there are two identities $e$ and $e'$. Then,\\
(a) $a * e = e * a = a, \quad \forall \,a \in G$\\
(b) $a * e' = e' * a = a, \quad \forall \,a \in G$\\
These two give us $e' * e = e * e' \Rightarrow \boxed{e' = e}$\\

NOTE: The notation of binary operation * will now be dropped, but it is always implied that
\begin{equation}
ab \equiv a*b
\end{equation}

\noindent
\textbf{Theorem 2.2}: In a group $G$, the Left and Right Cancellation Laws hold good.\\
\textbf{Proof}: Suppose $ba = ca$ and let $a'$ be the inverse of $a$. Then on multiplying on the right side, we get
\begin{align*}
(ba)a' &= (ca)a'\\
b(aa') &= c(aa') \quad \text{(Associativity)}\\
b &= c \hspace{1.25cm} \text{(Since $aa' = e = 1$)}
\end{align*}
This leads to another theorem that tells that the inverse of each element of a group is \textit{unique}.
\newpage
\textbf{Theorem 2.3}: For each element $a \in G$, where $G$ is a group, there exists a unique inverse $b$ such that $ab = ba = e$\\
\textbf{Proof}:Suppose $b$ and $c$ are inverses of $a \in G$. Then, 
\begin{align*}
ab = e, &\quad ac = e\\
ab &= ac\\
\Rightarrow b &= c \quad \text{(By Cancellation Laws)}
\end{align*}
A rather interesting result about the products and inverses of elements of a group is the \textbf{Socks-Shoes Property}:\\
\textbf{Theorem 2.4}: For group elements $a,b \in G$, $(ab)^{-1} = b^{-1}a^{-1}$\\
\textbf{Proof}: Consider
\begin{align*}
(ab)(ab)^{-1} &= e\\
(ab)(b^{-1}a^{-1}) &= e\\
a(b b^{-1})a^{-1} &= e\\
aa^{-1} &= e \Rightarrow e = e
\end{align*}
This result can be generalized as
\begin{equation}
(abc\ldots k)^{-1} = k^{-1} \ldots c^{-1}b^{-1} a^{-1}
\end{equation}
\textbf{Definition}: The \textit{order} of a group $G$ is the number of elements it has(finite or infinite), and is denoted as $|G|$\\
\textbf{Definition}: The \textit{order of an element} $g$ in a group $G$ is the smallest integer $n$ such that $g^n = e \in \,G$. If there is no such $n$, then the order of that element $|g|$ is said to be infinite. 

\subsubsection{Group Homomorphism}
\textbf{Definition}: A \textit{homomorphism} from a group $(G, \cdot)$ to another group $(G', *)$ is defined as
\begin{equation}
f(a \cdot b) = f(a) * f(b)
\end{equation}
A homomorphism has several kinds:
\begin{enumerate}
\item \textbf{Monomorphism} is a group homomorphism that is injective(one-to-one)
\item \textbf{Epimorphism} is a group homomorphism that is surjective(onto)
\item \textbf{Isomorphism} is a group homomorphism that is bijective(both one-to-one and onto). If this condition is satisfied, then for a homomorphism $f:G \to G'$, $G$ and $G'$ are said to be \textit{isomorphic groups}.
\item \textbf{Endomorphism} is a group homomorphism defined as $f:G \to G$, i.e., the same group $G$ is both codomain and range.
\item \textbf{Automorphism} is an endomorphism that is bijective, and hence an isomorphism.
\end{enumerate}
\textbf{Definition}: The \textit{kernel} of a homomorphism $h:G \to G'$ is defined as the set of elements of $G$ that map to the identity of $G'$
\begin{equation}
\text{ker}(h) = \{ k \in G \,| \,h(k) = e' \in G' \}
\end{equation}
\textbf{Definition}: The \textit{image} of the same homomorphism as above is defined as
\begin{equation}
\text{im}(h) = h(G) = \{ h(k) \,| \,k \in G \}
\end{equation}
\textbf{Fundamental Theorem of Homomorphism}\\
Given two groups $G$ and $G'$ and a group homomorphism defined as $f: G \to G'$, let $K$ be a normal subgroup in $G$, and $\phi: G \to G/K$. If $K$ is a subset of ker$(f)$, then there exists a unique homomorphism $h:G/K \to G'$ such that $f = h\phi$.

\subsubsection{Subgroup}
\textbf{Definition}: If a subset $H$ of a group $G$ is itself a group under the same binary operation of $G$, then $H$ is called a \textit{subgroup} of $G$. The identity of a subgroup is the same as the identity of the group, i.e. $e_H = e_G$. The inverse of an element of a subgroup is the same inverse of that element in that group. i.e. if $ab = ba = e_H$, then $ab = ba = e_G$\\\\
\textbf{Theorem 2.5 - The One-Step Subgroup Test}: Let $G$ be a group and $H$ be a non-empty subset of $G$. Then, $\forall \,a,b \in H$, if $ab^{-1} \in H$, then $H$ is a subgroup of $G$.\\
\textbf{Proof}: Let $a = x, \,b = x$ where $x \in H$. Then, $$ab^{-1} = xx^{-1} = e \in H$$
And if we choose $a = e$ and $b = x$, then
$$ab^{-1} \in H \Rightarrow ex^{-1} \in H \Rightarrow x^{-1} \in H$$
If some arbitrary $x, y \in H$, then $xy \in H$ since it is a subset of $G$. So, there is an identity element, inverse exists and since it is a subset of a group, associativity is satisfied. Therefore, $H$ is a subgroup of $G$.\\\\
\textbf{Theorem 2.6 - Two-Step Subgroup Test}: Let $G$ be a group and $H$ be a non-empty subset of $G$. We say $H$ is a subgroup of $G$ if 
\begin{align}
\text{1. $ab \in H, \forall \,a,b \in H$} \quad &\quad \quad \text{2. $a^{-1} \in H, \forall a \in H$}
\end{align}
The proof to Theorem 2.6 is left as an exercise to the reader. Note that if $H$ is closed under the same binary operation of $G$, even then, $H$ can be called a subgroup, since by default, $e \in H \Rightarrow a^{-1} \in H$. This is the Finite Subgroup Test.\\

\noindent
\textbf{Definition}: The \textit{center} $Z(G)$ of a group $G$ is a set of those elements that commutes with every other element of the group
\begin{equation}
Z(G) = \{ a \in G | \,ax = xa, \forall \,x \in G \}
\end{equation}
\newpage
\noindent
\textbf{Theorem 2.7}: The center $Z(G)$ of a group $G$ is a subgroup.\\
\textbf{Proof}: Clearly, $e \in Z(G) \Rightarrow Z(G) \neq \phi$. Take two elements $a, b \in Z(G)$. Then, 
$$(ab)x = a(bx) = a(xb) = (ax)b = (xa)b = x(ab) \Rightarrow ab \in Z(G) \quad \ldots (1)$$
Now consider,\\[-2.5em]
\begin{align*}
ax &= xa\\
a^{-1}(ax)a^{-1} &= a^{-1}(xa)a^{-1}\\
(a^{-1} a)xa^{-1} &= a^{-1}x (aa^{-1})\\
xa^{-1} &= a^{-1}x \Rightarrow a^{-1} \in Z(G) \quad \ldots (2)
\end{align*}
From (1) and (2) and Theorem 2.6, we can conclude that $Z(G)$ is a subgroup of $G$. From the definition of a center, the set of all such $x$ for a fixed $a \in G$ is called the \textit{centralizer} of $a$ in $G$
\begin{equation}
C(a) = \{ x \in G \,| ga = ag, \forall a \in G \}
\end{equation}
It can be proven similar to Theorem 2.7, that $C(a)$ is also a subgroup of $G$.

The intersection of any two subgroups $A$ and $B$ of a group $G$ is again a subgroup of $G$. The union of $A$ and $B$ is a subgroup iff either $A$ or $B$ contains the other.

\subsubsection{Cosets}
Let $H$ be a subgroup of a group $G$. Given $a \in G$, the \textbf{Left} and \textbf{Right} Cosets are obtained by multiplying each element of $H$ with a fixed element $a$ where $a$ is the left and the right factor respectively, i.e.,
\begin{subequations}
\begin{equation}
\text{Left Coset: } aH = \{ ah \,| \, a \in G, h \in H \}
\end{equation}
\begin{equation}
\text{Right Coset: } Ha = \{ ha \,| \, h \in H, a \in G \}
\end{equation}
\end{subequations}
If the group $G$ is Abelian, then the notation changes to $g + H$ and $H + g$ respectively. Properties of Cosets are as follows:
\begin{enumerate}
\item $a \in aH$\\
	\textbf{Proof:} $a = ae \in aH$
\item $aH = H$ iff $a \in H$\\
	\textbf{Proof:} Assume $a \in H$ and let $h \in H$. Then, since $a \in G$ and $h \in H$, we know $a^{-1}h \in H$. Then, $h = eh = (aa^{-1})h = a(a^{-1}h) \in H$ and therefore $H \subset aH$ By direct observation, $aH \subset H$. Hence, $aH = H$ iff $a \in H$
	
\item $aH = bH$ iff $a \in bH$\\
	\textbf{Proof:} If $aH = bH$, then $a = ae \in aH = bH$. Conversely, if $a \in bH \Rightarrow a = bh, \,h \in H$ and therefore $aH = b(hH) = bH$.
	
\item $aH = bH$ or $aH \cap bH = \phi$\\
	\textbf{Proof:} It follows from the previous property that if $\exists c \in (aH \cap bH)$, then $cH = aH$ and $cH = bH$
	
\item $aH = bH$ iff $a^{-1}b \in H$\\
	\textbf{Proof:} Notice that $aH = bH$ if and only if $H = a^{-1}bH$. From the second property, this property is fairly obvious.
	
\item $|aH| = |bH|$\\
	\textbf{Proof:} The correspondence $ah \to bh$ maps $aH \to bH$, and hence by cancellation laws, the one-to-one property follows.
	
\item $aH = Ha$ iff $H = aHa^{-1}$\\
	\textbf{Proof:} Notice that $aH = Ha$ iff $(aH)a^{-1} = (Ha)a^{-1} \Rightarrow H = aHa^{-1}$
	
\item $aH$ is subgroup of $G$ iff $a \in H$\\
	\textbf{Proof:} If $aH$ is a subgroup, then $e \in aH \Rightarrow aH \neq \phi$ and we have $aH = eH = H$. Thus from property 2, we have $a \in H$ and from its converse, we have that if $a \in H$, then again $aH = H$.
	
\end{enumerate}

\noindent
\textbf{Lagrange's Theorem}\\
If $G$ is a finite group and $H$ is a subgroup of $G$, then $|H|$ divides $|G|$ and the number of disctinct left(or right) cosets of $H$ in $G$ is $\frac{|G|}{|H|}$\\
\textbf{Proof}: Let $a_1H, a_2H, \ldots, a_r H$ be distinct left cosets of a subgroup $H$ in a group $G$. Then, $a \in G$, we have $aH = a_i H$ for some $i$. Then, the group is given as
$$ G = a_1H \cup a_2 H \cup \ldots \cup a_r H $$
and the order can then be written as
$$ |a_i H| = |H| \Rightarrow \boxed{|G| = r|H|} $$
Some Corollaries:
\begin{enumerate}
\item $|G:H| = |G|/|H|$
\item $|a|$ divides $|G|$
\item Groups whose order is a prime number are \textit{cyclic}
\item $a^{|G|} = e \in G$
\item \textbf{Fermat's Little Theorem}$$ a^p \,mod \,p = a \,mod \,p, \quad a \in \mathbb{Z}, \quad p = \text{prime number} $$
\end{enumerate}

\subsubsection{Normal Subgroups}
\textbf{Definition:} $\lhd$

\subsubsection{Quotient Groups}



\subsection{Semigroups and Monoids}

\subsection{Quasigroup and Loops}
\subsection{Abelian Group}
\subsection{Magma}
\subsection{Lie Group}
\subsection{Group Theory}

\newpage
\section{Ring Like}
\subsection{Ring}
\subsection{Semiring}
\subsection{Commutative Ring}
\subsection{Integral Domain}
\subsection{Fields}
\subsection{Ring Theory}

\newpage
\section{Lattice Like}
\subsection{Lattices}
\subsection{Semilattice}
\subsection{Boolean Algebra}
\subsection{Lattice Theory}

\newpage
\section{Module Like}
\subsection{Modules}
\subsection{Vector Space}
\subsection{Linear Algebra}

\newpage
\section{Algebra Like}
\subsection{Algebra}
\subsection{Associative and Non-Associative}
\subsection{Composition Algebra}
\subsection{Lie Algebra}
\subsection{Bialgebra}


\end{document}